\chapter{\Rp Code}


\section{{\tt llnorMmix}}

% it 1
Here {\tt llnorMmix}, since it is the central piece of the package,
and 2time.R as an example of a simulation script.


\begin{Schunk}
\begin{Soutput}
#### the llnorMmix function, calculating log likelihood for a given
#### parameter vector
## Author: Nicolas Trutmann 2019-07-06
## Log-likelihood of parameter vector given data
#
# par:   parameter vector
# tx:    transposed sample matrix
# k:     number of components
# model: assumed distribution model of normal mixture
# trafo: either centered log ratio or logit
llnorMmix <- function(par, tx, k,
                      trafo=c("clr1", "logit"),
                      model=c("EII","VII","EEI","VEI","EVI",
                              "VVI","EEE","VEE","EVV","VVV")
                      ) {
    stopifnot(is.matrix(tx),
              length(k <- as.integer(k)) == 1, k >= 1)
    p <- nrow(tx)
#    x <- t(x) ## then only needed in   (x-mu[,i])^2  i=1..k
    # 2. transform
    model <- match.arg(model)
    trafo <- match.arg(trafo)
    l2pi <- log(2*pi)
    # 3. calc log-lik
    # get w
    w <- if (k==1) 1
         else switch(trafo,
                     "clr1" = clr1inv (par[1:(k-1)]),
                     "logit"= logitinv(par[1:(k-1)]),
                     stop("invalid 'trafo': ", trafo)
         )
    # start of relevant parameters:
    f <- k + p*k # weights -1 + means +1 => start of alpha
    # get mu
    mu <- matrix(par[k:(f-1L)], p,k)
    f1 <- f      # end of alpha if uniform
    f2 <- f+k-1L # end of alpha if var
    f1.1 <- f1 +1L # start of D. if alpha unif.
    f2.1 <- f1 + k # start of D. if alpha variable
    f11 <- f1 + p-1    # end of D. if D. uniform and alpha uniform
    f12 <- f1 +(p-1)*k # end    D. if D.   var   and alpha unif.
    f21 <- f2 + p-1    # end of D. if D. uniform and alpha variable
    f22 <- f2 +(p-1)*k # end of D. if D.   var   and alpha var.
    f11.1 <- f11 +1L # start of L if alpha unif  D unif
    f21.1 <- f21 +1L # start of L if alpha var   D unif
    f12.1 <- f12 +1L # start of L if alpha unif  D var
    f22.1 <- f22 +1L # start of L if alpha var   D var
    f111 <- f11 +   p*(p-1)/2 # end of L if alpha unif  D unif
    f211 <- f21 +   p*(p-1)/2 # end of L if alpha var   D unif
    f121 <- f12 + k*p*(p-1)/2 # end of L if alpha unif  D var
    f221 <- f22 + k*p*(p-1)/2 # end of L if alpha var   D var
    # initialize f(tx_i) i=1..n  vector of density values
    invl <- 0
    # calculate log-lik, see first case for explanation
    switch(model,
    "EII" = {
        alpha <- par[f]
        invalpha <- exp(-alpha)# = 1/exp(alpha)
        for (i in 1:k) {
            rss <- colSums(invalpha*(tx-mu[,i])^2)
            # this is vector of length n=sample size
            # calculates (tx-mu)t * Sigma^-1 * (tx-mu) for diagonal
            # cases.
            invl <- invl+w[i]*exp(-0.5*(p*(alpha+l2pi)+rss))
            # adds likelihood of one component to invl
            # the formula in exp() is the log of likelihood
            # still of length n
        }
    },
    # hereafter differences are difference in dimension in alpha and D.
    # alpha / alpha[i] and D. / D.[,i]
    "VII" = {
        alpha <- par[f:f2]
        for (i in 1:k) {
            rss <- colSums((tx-mu[,i])^2/exp(alpha[i]))
            invl <- invl+w[i]*exp(-0.5*(p*(alpha[i]+l2pi)+rss))
        }
    },
    "EEI" = {
        alpha <- par[f]
        D. <- par[f1.1:f11]
        D. <- c(-sum(D.),D.)
        D. <- D.-sum(D.)/p
        invD <- exp(alpha+D.)
        for (i in 1:k) {
            rss <- colSums((tx-mu[,i])^2/invD)
            invl <- invl+w[i]*exp(-0.5*(p*(alpha+l2pi)+rss))
        }
    },
    "VEI" = {
        alpha <- par[f:f2]
        D. <- par[f2.1:f21]
        D. <- c(-sum(D.), D.)
        D. <- D.-sum(D.)/p
        for (i in 1:k) {
            rss <- colSums((tx-mu[,i])^2/exp(alpha[i]+D.))
            invl <- invl+w[i]*exp(-0.5*(p*(alpha[i]+l2pi)+rss))
        }
    },
    "EVI" = {
        alpha <- par[f]
        D. <- matrix(par[f1.1:f12],p-1,k)
        D. <- apply(D.,2, function(j) c(-sum(j), j))
        D. <- apply(D.,2, function(j) j-sum(j)/p)
        for (i in 1:k) {
            rss <- colSums((tx-mu[,i])^2/exp(alpha+D.[,i]))
            invl <- invl+w[i]*exp(-0.5*(p*(alpha+l2pi)+rss))
        }
    },
    "VVI" = {
        alpha <- par[f:f2]
        D. <- matrix(par[f2.1:f22],p-1,k)
        D. <- apply(D.,2, function(j) c(-sum(j), j))
        D. <- apply(D.,2, function(j) j-sum(j)/p)
        for (i in 1:k) {
            rss <- colSums((tx-mu[,i])^2/exp(alpha[i]+D.[,i]))
            invl <- invl+w[i]*exp(-0.5*(p*(alpha[i]+l2pi)+rss))
        }
    },
    # here start the non-diagonal cases. main difference is the use
    # of backsolve() to calculate tx^t Sigma^-1 tx, works as follows:
    # assume Sigma = L D L^t, then Sigma^-1 = (L^t)^-1 D^-1 L^-1
    # y = L^-1 tx  => tx^t Sigma^-1 tx = y^t D^-1 y
    # y = backsolve(L., tx)
    "EEE" = {
        alpha <- par[f]
        D. <- par[f1.1:f11]
        D. <- c(-sum(D.), D.)
        D. <- D.-sum(D./p)
        invD <- exp(alpha+D.)
        L. <- diag(1,p)
        L.[lower.tri(L., diag=FALSE)] <- par[f11.1:f111]
        for (i in 1:k) {
            rss <- colSums(backsolve(L.,(tx-mu[,i]), upper.tri=FALSE)^2/invD)
            invl <- invl+w[i]*exp(-0.5*(p*(alpha+l2pi)+rss))
        }
    },
    "VEE" = {
        alpha <- par[f:f2]
        D. <- par[f2.1:f21]
        D. <- c(-sum(D.), D.)
        D. <- D.-sum(D./p)
        L. <- diag(1,p)
        L.[lower.tri(L., diag=FALSE)] <- par[f21.1:f211]
        for (i in 1:k) {
            rss <- colSums(backsolve(L., (tx-mu[,i]), upper.tri=FALSE)^2/exp(alpha[i]+D.))
            invl <- invl+w[i]*exp(-0.5*(p*(alpha[i]+l2pi)+rss))
        }
    },
    "EVV" = {
        alpha <- par[f]
        D. <- matrix(par[f1.1:f12],p-1,k)
        D. <- apply(D.,2, function(j) c(-sum(j), j))
        D. <- apply(D.,2, function(j) j-sum(j)/p)
        L.temp <- matrix(par[f12.1:f121],p*(p-1)/2,k)
        for (i in 1:k) {
            L. <- diag(1,p)
            L.[lower.tri(L., diag=FALSE)] <- L.temp[,i]
            rss <- colSums(backsolve(L., (tx-mu[,i]), upper.tri=FALSE)^2/exp(alpha+D.[,i]))
            invl <- invl+w[i]*exp(-0.5*(p*(alpha+l2pi)+rss))
        }
    },
    "VVV" = {
        alpha <- par[f:f2]
        D. <- matrix(par[f2.1:f22],p-1,k)
        D. <- apply(D.,2, function(j) c(-sum(j), j))
        D. <- apply(D.,2, function(j) j-sum(j)/p)
        invalpha <- exp(rep(alpha, each=p)+D.)
        L.temp <- matrix(par[f22.1:f221],p*(p-1)/2,k)
        L. <- diag(1,p)
        for (i in 1:k) {
            L.[lower.tri(L., diag=FALSE)] <- L.temp[,i]
            rss <- colSums(backsolve(L., (tx-mu[,i]), upper.tri=FALSE)^2/invalpha[,i])
            invl <- invl+w[i]*exp(-0.5*(p*(alpha[i]+l2pi)+rss))
        }
    },
    ## otherwise
    stop("invalid model:", model)
    )
    ## return  sum_{i=1}^n log( f(tx_i) ) :
    sum(log(invl))
}
sllnorMmix <- function(x, obj, trafo=c("clr1", "logit")) {
    stopifnot(is.character(model <- obj$model))
    trafo <- match.arg(trafo)
    llnorMmix(nMm2par(obj, model=model),
              tx = t(x), k = obj$k, 
              model=model, trafo=trafo)
}
## log-likelihood function relying on mvtnorm function
#
# par:   parameter vector as calculated by nMm2par
# x:     matrix of samples
# k:     number of cluster
# trafo: transformation of weights
# model: assumed model of the distribution
llmvtnorm <- function(par, x, k,
                      trafo=c("clr1", "logit"),
                      model=c("EII","VII","EEI","VEI","EVI",
                              "VVI","EEE","VEE","EVV","VVV")
              ) {
    stopifnot(is.matrix(x),
              length(k <- as.integer(k)) == 1, k >= 1)
    model <- match.arg(model)
    trafo <- match.arg(trafo)
    p <- ncol(x)
    nmm <- par2nMm(par, p, k, model=model, trafo=trafo)
    ## FIXME (speed!):  dmvnorm(*, sigma= S) will do a chol(S) for each component
    ## -----  *instead* we already have LDL' and  chol(S) = sqrt(D) L' !!
    ## another par2*() function should give L and D, or from that chol(Sagma), rather than Sigma !
    w <- nmm$w
    mu <- nmm$mu
    sig <- nmm$Sigma
    y <- 0
    for (i in 1:k) {
        y <- y + w[i]*mvtnorm::dmvnorm(x,mean=mu[,i],sigma=sig[,,i])
    }
    sum(log(y))
}
\end{Soutput}
\end{Schunk}

\clearpage
\section{Example Simulation Script}

\label{App:sims}
% TODO:
here e.g. 2init.R and write some remarks on it.

\begin{Schunk}
\begin{Soutput}
## Intent: analyse time as function of p,k,n
nmmdir <- normalizePath("~/BachelorArbeit/norMmix.Rcheck/")
savdir <- normalizePath("~/BachelorArbeit/Rscripts/2time")
stopifnot(dir.exists(nmmdir), dir.exists(savdir))
library(norMmix, lib.loc=nmmdir)
library(mclust)
## at n=500,p=2 can do about 250xfitnMm(x,1:10) in 24h
seeds <- 1:10
sizes <- c(500, 1000, 2000)
nmm <- list(MW214, MW34, MW51)
## => about 100 cases
# for naming purposes
nmnames <- c("MW214", "MW34", "MW51")
sizenames <- c("500", "1000", "2000")
files <- vector(mode="character")
for (nm in 1:3) {
    for (size in sizes) {
    set.seed(2019); x <- rnorMmix(size, nmm[[nm]])
        for (seed in seeds) {
            set.seed(2019+seed)
            r <- tryCatch(fitnMm(x, k=1:8,
                                 optREPORT=1e4, maxit=1e4),
                          error = identity)
            filename <- sprintf("%s_size=%0.4d_seed=%0.2d.rds",
                                nmnames[nm], size, seed)
            files <- append(files, filename)
            cat("===> saving to file:", filename, "\n")
            saveRDS(list(fit=r), file=file.path(savdir, filename))
        }
    }
}
fillis <- list()
for (i in seq_along(sizes)) {
    for (j in seq_along(nmnames)) {
        # for lack of AND matching, OR match everything else and invert
        ret <- grep(paste(sizenames[-i], nmnames[-j], sep="|"), 
                    files, value=TRUE, invert=TRUE)
        fillis[[paste0(sizenames[i], nmnames[j])]] <- ret
    }
}
epfl(fillis, savdir)
\end{Soutput}
\end{Schunk}

